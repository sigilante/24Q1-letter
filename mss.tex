\documentclass[twoside]{article}

\usepackage{ustj}

\newcommand{\authorname}{N. E. Davis}
\newcommand{\authorpatp}{\patp{lagrev-nocfep}}
\newcommand{\affiliation}{Urbit Foundation}

%  Make first page footer:
\fancypagestyle{firststyle}{%
\fancyhf{}% Clear header/footer
\fancyhead{}
\fancyfoot[L]{{\footnotesize
              %% We toggle between these:
              Manuscript submitted for review.\\
              % {\it Urbit Systems Technical Journal} I:1 (2024):  1–3. \\
              ~ \\
              Address author correspondence to \authorpatp.
              }}
}
%  Arrange subsequent pages:
\fancyhf{}
\fancyhead[LE]{{\urbitfont Urbit Systems Technical Journal}}
\fancyhead[RO]{Letter from the Editor}
\fancyfoot[LE,RO]{\thepage}

%%MANUSCRIPT
\title{Letter from the Editor}
\author{\authorname~\authorpatp \\ \affiliation}
\date{}

\begin{document}

\maketitle
\thispagestyle{firststyle}

% We will adjust page numbering in final editing.
\pagenumbering{arabic}
\setcounter{page}{1}

It never felt like the timing was right for producing and releasing a reference book or textbook:  Urbit was too hot (in the kelvin sense) and some of the prodigious work of producing a volume would have to be repeated for every system release.  There are deep and true things that can be said about the platonic Urbit, the diamond Urbit, but much of what we are working through now is the contingent Urbit, still feeling our way towards zero kelvin.  A technical journal is more forgiving in all the senses we want:  it is time-resolved; it is episodic; and it allows deep rabbit holes that would never fit in a book.  Rather than settle for the compromises of a textbook, we can instead find ways to start saying every important technical thing about Urbit and solid-state computing.

Note “Urbit System\emph{s} Technical Journal”.  We recognize and encourage that deterministic computing, secure computing, and TODO are all larger fields than Urbit alone.

USTJ will follow a permissive dictum:  “Therefore every scribe which is instructed unto the kingdom of heaven is like unto a man that is an householder, which bringeth forth out of his treasure things new and old” (Matthew xiii 52, KJV).  In these pages you will read groundbreaking new work as well as well-considered expositions from Urbit's development.

Exit the only game in town.

I would particularly like to thank \patp{mopfel-winrux}~for his early enthusiasm for the concept of USTJ; \patp{wolref-podlex}~for paving the 
each contributor, reviewer, and designer
Simon DeDeo 
Stuart

When I first encountered Urbit, it was like Mr Electrico touching twelve-year-old Ray Bradbury with a charged sword and commanding him to “live forever!”

\end{document}
